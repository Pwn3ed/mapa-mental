\documentclass[a4paper,12pt]{article}
\usepackage[utf8]{inputenc}
\usepackage[portuguese]{babel}
\usepackage[T1]{fontenc}
\usepackage{tikz}
\usetikzlibrary{shapes,positioning,backgrounds}
\pgfdeclarelayer{background}
\pgfsetlayers{background,main}

\usepackage{geometry}
\geometry{paperheight=300mm, paperwidth=350mm, margin=1.5cm}

\title{Mapa Mental: Redes, Sociedade e Privacidade}
\author{Leonardo Luz e Diego Prestes}
\date{\today}

\begin{document}
\maketitle

\vspace{2cm}

\begin{center}
\begin{tikzpicture}[every node/.style={draw, thick, rounded corners, align=center, minimum width=30mm, minimum height=12mm, fill=blue!15, font=\sffamily}]

% ---------------- Central Node ----------------
\node[fill=red!20, minimum width=40mm, minimum height=25mm] (centro) {Redes digitais \\ e sociedade};

% ---------------- Dilema das Redes Branch ----------------
\node[fill=orange!40, right=6cm of centro, yshift=-4cm] (dilema) {O Dilema das Redes};
\node[fill=orange!20, above=2.0cm of dilema] (dilema1) {Manipulação algorítmica};
\node[fill=orange!20, above=0.6cm of dilema] (dilema2) {Economia da atenção};
\node[fill=orange!20, below=0.6cm of dilema] (dilema3) {Impactos:\\polarização, vício};
\node[fill=orange!20, below=2.0cm of dilema] (dilema4) {Reflexão:\\Preocupante — ética negligenciada};

% ---------------- Castells Branch ----------------
\node[fill=green!40, above=4cm of centro] (castells) {Castells:\\Sociedade em Rede};
\node[fill=green!20, above=0.2cm of castells] (castells1) {Estrutura em rede:\\flexível, global};
\node[fill=green!20, right=0.2cm of castells] (castells2) {Transformações:\\trabalho, cultura, política};
\node[fill=green!20, left=0.2cm of castells] (castells3) {Informação como poder};
\node[fill=green!20, below=0.2cm of castells] (castells4) {Reflexão:\\Transformação inevitável, mas desigual};

% ---------------- Privacidade Hackeada Branch ----------------
\node[fill=purple!40, left=6cm of centro, yshift=-4cm] (privacidade) {Privacidade Hackeada};
\node[fill=purple!20, above=2.0cm of privacidade] (priv1) {Cambridge Analytica};
\node[fill=purple!20, above=0.6cm of privacidade] (priv2) {Vigilância digital e uso político dos dados};
\node[fill=purple!20, below=0.6cm of privacidade] (priv3) {Exemplos:\\Brexit, Trump 2016};
\node[fill=purple!20, below=2.0cm of privacidade] (priv4) {Reflexão:\\Urgência de regulação e ética profissional};

% ---------------- Connections ----------------
\begin{scope}[on background layer]
  \draw[->, thick] (castells) -- (dilema)      node[midway, above] {Lógica em rede potencializa algoritmos};
  \draw[->, thick] (dilema)   -- (privacidade) node[midway] {Dados pessoais alimentam modelos de manipulação};
  \draw[->, thick] (castells) -- (privacidade) node[midway, above] {Redes globais favorecem coleta massiva};
\end{scope}

\end{tikzpicture}
\end{center}

\end{document}
